
%% bare_conf.tex
%% V1.3
%% 2007/01/11
%% by Michael Shell
%% See:
%% http://www.michaelshell.org/
%% for current contact information.
%%
%% This is a skeleton file demonstrating the use of IEEEtran.cls
%% (requires IEEEtran.cls version 1.7 or later) with an IEEE conference paper.
%%
%% Support sites:
%% http://www.michaelshell.org/tex/ieeetran/
%% http://www.ctan.org/tex-archive/macros/latex/contrib/IEEEtran/
%% and
%% http://www.ieee.org/

%%*************************************************************************
%% Legal Notice:
%% This code is offered as-is without any warranty either expressed or
%% implied; without even the implied warranty of MERCHANTABILITY or
%% FITNESS FOR A PARTICULAR PURPOSE! 
%% User assumes all risk.
%% In no event shall IEEE or any contributor to this code be liable for
%% any damages or losses, including, but not limited to, incidental,
%% consequential, or any other damages, resulting from the use or misuse
%% of any information contained here.
%%
%% All comments are the opinions of their respective authors and are not
%% necessarily endorsed by the IEEE.
%%
%% This work is distributed under the LaTeX Project Public License (LPPL)
%% ( http://www.latex-project.org/ ) version 1.3, and may be freely used,
%% distributed and modified. A copy of the LPPL, version 1.3, is included
%% in the base LaTeX documentation of all distributions of LaTeX released
%% 2003/12/01 or later.
%% Retain all contribution notices and credits.
%% ** Modified files should be clearly indicated as such, including  **
%% ** renaming them and changing author support contact information. **
%%
%% File list of work: IEEEtran.cls, IEEEtran_HOWTO.pdf, bare_adv.tex,
%%                    bare_conf.tex, bare_jrnl.tex, bare_jrnl_compsoc.tex
%%*************************************************************************

% *** Authors should verify (and, if needed, correct) their LaTeX system  ***
% *** with the testflow diagnostic prior to trusting their LaTeX platform ***
% *** with production work. IEEE's font choices can trigger bugs that do  ***
% *** not appear when using other class files.                            ***
% The testflow support page is at:
% http://www.michaelshell.org/tex/testflow/



% Note that the a4paper option is mainly intended so that authors in
% countries using A4 can easily print to A4 and see how their papers will
% look in print - the typesetting of the document will not typically be
% affected with changes in paper size (but the bottom and side margins will).
% Use the testflow package mentioned above to verify correct handling of
% both paper sizes by the user's LaTeX system.
%
% Also note that the "draftcls" or "draftclsnofoot", not "draft", option
% should be used if it is desired that the figures are to be displayed in
% draft mode.
%
\documentclass[conference]{IEEEtran}
% Add the compsoc option for Computer Society conferences.
%
% If IEEEtran.cls has not been installed into the LaTeX system files,
% manually specify the path to it like:
% \documentclass[conference]{../sty/IEEEtran}





% Some very useful LaTeX packages include:
% (uncomment the ones you want to load)


% *** MISC UTILITY PACKAGES ***
%
%\usepackage{ifpdf}
% Heiko Oberdiek's ifpdf.sty is very useful if you need conditional
% compilation based on whether the output is pdf or dvi.
% usage:
% \ifpdf
%   % pdf code
% \else
%   % dvi code
% \fi
% The latest version of ifpdf.sty can be obtained from:
% http://www.ctan.org/tex-archive/macros/latex/contrib/oberdiek/
% Also, note that IEEEtran.cls V1.7 and later provides a builtin
% \ifCLASSINFOpdf conditional that works the same way.
% When switching from latex to pdflatex and vice-versa, the compiler may
% have to be run twice to clear warning/error messages.






% *** CITATION PACKAGES ***
%
%\usepackage{cite}
% cite.sty was written by Donald Arseneau
% V1.6 and later of IEEEtran pre-defines the format of the cite.sty package
% \cite{} output to follow that of IEEE. Loading the cite package will
% result in citation numbers being automatically sorted and properly
% "compressed/ranged". e.g., [1], [9], [2], [7], [5], [6] without using
% cite.sty will become [1], [2], [5]--[7], [9] using cite.sty. cite.sty's
% \cite will automatically add leading space, if needed. Use cite.sty's
% noadjust option (cite.sty V3.8 and later) if you want to turn this off.
% cite.sty is already installed on most LaTeX systems. Be sure and use
% version 4.0 (2003-05-27) and later if using hyperref.sty. cite.sty does
% not currently provide for hyperlinked citations.
% The latest version can be obtained at:
% http://www.ctan.org/tex-archive/macros/latex/contrib/cite/
% The documentation is contained in the cite.sty file itself.






% *** GRAPHICS RELATED PACKAGES ***
%
\ifCLASSINFOpdf
  % \usepackage[pdftex]{graphicx}
  % declare the path(s) where your graphic files are
  % \graphicspath{{../pdf/}{../jpeg/}}
  % and their extensions so you won't have to specify these with
  % every instance of \includegraphics
  % \DeclareGraphicsExtensions{.pdf,.jpeg,.png}
\else
  % or other class option (dvipsone, dvipdf, if not using dvips). graphicx
  % will default to the driver specified in the system graphics.cfg if no
  % driver is specified.
  % \usepackage[dvips]{graphicx}
  % declare the path(s) where your graphic files are
  % \graphicspath{{../eps/}}
  % and their extensions so you won't have to specify these with
  % every instance of \includegraphics
  % \DeclareGraphicsExtensions{.eps}
\fi
% graphicx was written by David Carlisle and Sebastian Rahtz. It is
% required if you want graphics, photos, etc. graphicx.sty is already
% installed on most LaTeX systems. The latest version and documentation can
% be obtained at: 
% http://www.ctan.org/tex-archive/macros/latex/required/graphics/
% Another good source of documentation is "Using Imported Graphics in
% LaTeX2e" by Keith Reckdahl which can be found as epslatex.ps or
% epslatex.pdf at: http://www.ctan.org/tex-archive/info/
%
% latex, and pdflatex in dvi mode, support graphics in encapsulated
% postscript (.eps) format. pdflatex in pdf mode supports graphics
% in .pdf, .jpeg, .png and .mps (metapost) formats. Users should ensure
% that all non-photo figures use a vector format (.eps, .pdf, .mps) and
% not a bitmapped formats (.jpeg, .png). IEEE frowns on bitmapped formats
% which can result in "jaggedy"/blurry rendering of lines and letters as
% well as large increases in file sizes.
%
% You can find documentation about the pdfTeX application at:
% http://www.tug.org/applications/pdftex





% *** MATH PACKAGES ***
%
%\usepackage[cmex10]{amsmath}
% A popular package from the American Mathematical Society that provides
% many useful and powerful commands for dealing with mathematics. If using
% it, be sure to load this package with the cmex10 option to ensure that
% only type 1 fonts will utilized at all point sizes. Without this option,
% it is possible that some math symbols, particularly those within
% footnotes, will be rendered in bitmap form which will result in a
% document that can not be IEEE Xplore compliant!
%
% Also, note that the amsmath package sets \interdisplaylinepenalty to 10000
% thus preventing page breaks from occurring within multiline equations. Use:
%\interdisplaylinepenalty=2500
% after loading amsmath to restore such page breaks as IEEEtran.cls normally
% does. amsmath.sty is already installed on most LaTeX systems. The latest
% version and documentation can be obtained at:
% http://www.ctan.org/tex-archive/macros/latex/required/amslatex/math/





% *** SPECIALIZED LIST PACKAGES ***
%
%\usepackage{algorithmic}
% algorithmic.sty was written by Peter Williams and Rogerio Brito.
% This package provides an algorithmic environment fo describing algorithms.
% You can use the algorithmic environment in-text or within a figure
% environment to provide for a floating algorithm. Do NOT use the algorithm
% floating environment provided by algorithm.sty (by the same authors) or
% algorithm2e.sty (by Christophe Fiorio) as IEEE does not use dedicated
% algorithm float types and packages that provide these will not provide
% correct IEEE style captions. The latest version and documentation of
% algorithmic.sty can be obtained at:
% http://www.ctan.org/tex-archive/macros/latex/contrib/algorithms/
% There is also a support site at:
% http://algorithms.berlios.de/index.html
% Also of interest may be the (relatively newer and more customizable)
% algorithmicx.sty package by Szasz Janos:
% http://www.ctan.org/tex-archive/macros/latex/contrib/algorithmicx/




% *** ALIGNMENT PACKAGES ***
%
%\usepackage{array}
% Frank Mittelbach's and David Carlisle's array.sty patches and improves
% the standard LaTeX2e array and tabular environments to provide better
% appearance and additional user controls. As the default LaTeX2e table
% generation code is lacking to the point of almost being broken with
% respect to the quality of the end results, all users are strongly
% advised to use an enhanced (at the very least that provided by array.sty)
% set of table tools. array.sty is already installed on most systems. The
% latest version and documentation can be obtained at:
% http://www.ctan.org/tex-archive/macros/latex/required/tools/


%\usepackage{mdwmath}
%\usepackage{mdwtab}
% Also highly recommended is Mark Wooding's extremely powerful MDW tools,
% especially mdwmath.sty and mdwtab.sty which are used to format equations
% and tables, respectively. The MDWtools set is already installed on most
% LaTeX systems. The lastest version and documentation is available at:
% http://www.ctan.org/tex-archive/macros/latex/contrib/mdwtools/


% IEEEtran contains the IEEEeqnarray family of commands that can be used to
% generate multiline equations as well as matrices, tables, etc., of high
% quality.


%\usepackage{eqparbox}
% Also of notable interest is Scott Pakin's eqparbox package for creating
% (automatically sized) equal width boxes - aka "natural width parboxes".
% Available at:
% http://www.ctan.org/tex-archive/macros/latex/contrib/eqparbox/





% *** SUBFIGURE PACKAGES ***
%\usepackage[tight,footnotesize]{subfigure}
% subfigure.sty was written by Steven Douglas Cochran. This package makes it
% easy to put subfigures in your figures. e.g., "Figure 1a and 1b". For IEEE
% work, it is a good idea to load it with the tight package option to reduce
% the amount of white space around the subfigures. subfigure.sty is already
% installed on most LaTeX systems. The latest version and documentation can
% be obtained at:
% http://www.ctan.org/tex-archive/obsolete/macros/latex/contrib/subfigure/
% subfigure.sty has been superceeded by subfig.sty.



%\usepackage[caption=false]{caption}
%\usepackage[font=footnotesize]{subfig}
% subfig.sty, also written by Steven Douglas Cochran, is the modern
% replacement for subfigure.sty. However, subfig.sty requires and
% automatically loads Axel Sommerfeldt's caption.sty which will override
% IEEEtran.cls handling of captions and this will result in nonIEEE style
% figure/table captions. To prevent this problem, be sure and preload
% caption.sty with its "caption=false" package option. This is will preserve
% IEEEtran.cls handing of captions. Version 1.3 (2005/06/28) and later 
% (recommended due to many improvements over 1.2) of subfig.sty supports
% the caption=false option directly:
%\usepackage[caption=false,font=footnotesize]{subfig}
%
% The latest version and documentation can be obtained at:
% http://www.ctan.org/tex-archive/macros/latex/contrib/subfig/
% The latest version and documentation of caption.sty can be obtained at:
% http://www.ctan.org/tex-archive/macros/latex/contrib/caption/




% *** FLOAT PACKAGES ***
%
%\usepackage{fixltx2e}
% fixltx2e, the successor to the earlier fix2col.sty, was written by
% Frank Mittelbach and David Carlisle. This package corrects a few problems
% in the LaTeX2e kernel, the most notable of which is that in current
% LaTeX2e releases, the ordering of single and double column floats is not
% guaranteed to be preserved. Thus, an unpatched LaTeX2e can allow a
% single column figure to be placed prior to an earlier double column
% figure. The latest version and documentation can be found at:
% http://www.ctan.org/tex-archive/macros/latex/base/



%\usepackage{stfloats}
% stfloats.sty was written by Sigitas Tolusis. This package gives LaTeX2e
% the ability to do double column floats at the bottom of the page as well
% as the top. (e.g., "\begin{figure*}[!b]" is not normally possible in
% LaTeX2e). It also provides a command:
%\fnbelowfloat
% to enable the placement of footnotes below bottom floats (the standard
% LaTeX2e kernel puts them above bottom floats). This is an invasive package
% which rewrites many portions of the LaTeX2e float routines. It may not work
% with other packages that modify the LaTeX2e float routines. The latest
% version and documentation can be obtained at:
% http://www.ctan.org/tex-archive/macros/latex/contrib/sttools/
% Documentation is contained in the stfloats.sty comments as well as in the
% presfull.pdf file. Do not use the stfloats baselinefloat ability as IEEE
% does not allow \baselineskip to stretch. Authors submitting work to the
% IEEE should note that IEEE rarely uses double column equations and
% that authors should try to avoid such use. Do not be tempted to use the
% cuted.sty or midfloat.sty packages (also by Sigitas Tolusis) as IEEE does
% not format its papers in such ways.





% *** PDF, URL AND HYPERLINK PACKAGES ***
%
%\usepackage{url}
% url.sty was written by Donald Arseneau. It provides better support for
% handling and breaking URLs. url.sty is already installed on most LaTeX
% systems. The latest version can be obtained at:
% http://www.ctan.org/tex-archive/macros/latex/contrib/misc/
% Read the url.sty source comments for usage information. Basically,
% \url{my_url_here}.





% *** Do not adjust lengths that control margins, column widths, etc. ***
% *** Do not use packages that alter fonts (such as pslatex).         ***
% There should be no need to do such things with IEEEtran.cls V1.6 and later.
% (Unless specifically asked to do so by the journal or conference you plan
% to submit to, of course. )



\usepackage[pdftex]{graphicx}


% correct bad hyphenation here
\hyphenation{op-tical net-works semi-conduc-tor}

\begin{document}
%
% paper title
% can use linebreaks \\ within to get better formatting as desired
\normalsize
\title{Smart Brochure \\ (paperless brochure using beacon)}


% author names and affiliations
% use a multiple column layout for up to three different
% affiliations
\author{\IEEEauthorblockN{Hyeonsu Lim}
\IEEEauthorblockA{Information System in HYU\\
Email: tatto\_hs@naver.com }
\and
\IEEEauthorblockN{Jaemook Kang}
\IEEEauthorblockA{Information System in HYU \\
Email: kjm8475@naver.com}
\and
\IEEEauthorblockN{Kiseong Kim\\Information System in HYU\\
Email: rltjd1231@naver.com \\\\\\\\\\\\\\}}

% conference papers do not typically use \thanks and this command
% is locked out in conference mode. If really needed, such as for
% the acknowledgment of grants, issue a \IEEEoverridecommandlockouts
% after \documentclass

% for over three affiliations, or if they all won't fit within the width
% of the page, use this alternative format:
% 
%\author{\IEEEauthorblockN{Michael Shell\IEEEauthorrefmark{1},
%Homer Simpson\IEEEauthorrefmark{2},
%James Kirk\IEEEauthorrefmark{3}, 
%Montgomery Scott\IEEEauthorrefmark{3} and
%Eldon Tyrell\IEEEauthorrefmark{4}}
%\IEEEauthorblockA{\IEEEauthorrefmark{1}School of Electrical and Computer Engineering\\
%Georgia Institute of Technology,
%Atlanta, Georgia 30332--0250\\ Email: see http://www.michaelshell.org/contact.html}
%\IEEEauthorblockA{\IEEEauthorrefmark{2}Twentieth Century Fox, Springfield, USA\\
%Email: homer@thesimpsons.com}
%\IEEEauthorblockA{\IEEEauthorrefmark{3}Starfleet Academy, San Francisco, California 96678-2391\\
%Telephone: (800) 555--1212, Fax: (888) 555--1212}
%\IEEEauthorblockA{\IEEEauthorrefmark{4}Tyrell Inc., 123 Replicant Street, Los Angeles, California 90210--4321}}




% use for special paper notices
%\IEEEspecialpapernotice{(Invited Paper)}




% make the title area
\maketitle


\begin{abstract}
%\boldmath
When you go to the art museum, you will find out several things to help you see the exhibition better, such as brochure, program books, and audio-guide, etc. For the big size of exhibitions supported by big art gallery or of famous artist, there would be no problem to prepare the goods mentioned before. However, there are a lot of artists who are trying to open an exhibition in small art gallery and students who are preparing the exhibition for the graduation, and they have a lot of problems to possess those kinds of goods.
The purpose of this software project is to help them. We can provide many kinds of IoT services, such as the explanation of the exhibition, explanations of each art, audio-guide, and so on, by using beacon and the mobile application.\\

Keywords 
---
beacon; bluetooth; iot; museum; brochure; smart phone; application; \\\\\\\\\\

\end{abstract}


\renewcommand{\arrayrulewidth}{1pt}

\begin{tabular}{|l|l|l|}\hline
 Roles& Name & Task description and etc. \\\hline\hline
User & Kiseong Kim & Use the mobile application and get some data about exhibition \\
Customer & Kiseong Kim & Purchase this software service and offer the data about exhibition \\
Software Developer & Jaemook Kang & Develop the mobile application and back-end server\\
Development manager & Hyeonsu Lim &  Manage team and project and make the every plan of the process of project\\\hline

\end{tabular}
\\\\\\\\\\\\\\\\\\\\\\\\\\\\\\\\\\\\\\\\\\\\\\\\\\\\\\\\\\\\\\\\\\\\\\\\\\\\\\\\\\\\\\\\\\\\\\\\\\\\\\\\\

% IEEEtran.cls defaults to using nonbold math in the Abstract.
% This preserves the distinction between vectors and scalars. However,
% if the conference you are submitting to favors bold math in the abstract,
% then you can use LaTeX's standard command \boldmath at the very start
% of the abstract to achieve this. Many IEEE journals/conferences frown on
% math in the abstract anyway.

% no keywords




% For peer review papers, you can put extra information on the cover
% page as needed:
% \ifCLASSOPTIONpeerreview
% \begin{center} \bfseries EDICS Category: 3-BBND \end{center}
% \fi
%
% For peerreview papers, this IEEEtran command inserts a page break and
% creates the second title. It will be ignored for other modes.
\IEEEpeerreviewmaketitle


\large
\section{Introduction}
% no \IEEEPARstart
If you have only a little interest, you will find many art galleries and museums easily. There are over 100 exhibition halls in Seoul, which means that a lot of exhibitions we can enjoy are opening every day. In order to help the audiences enjoy these exhibitions better and feel better, those exhibitions are providing many items that will help people understand the exhibition better. The audiences need to pay extra costs to buy or rent those goods.

However, not all the exhibitions provide those services. In the case of famous artists and big art museums, a lot of people will visit there and see the works, so there will be many people who will pay extra costs to get the chance to inspect better. Therefore, famous artists and big art museums can make extra incomes by making lots of guide goods and sell them.
On the other hand, let’s think about obscure artists and the university students who are preparing the exhibition for the graduations. They can hardly provide those items for their audiences. The first reason is the cost. The cost to make audio guide, brochure, and program books are not low, so the obscure artists and the students cannot afford it.(about 2,000,000 won)

 The second reason is the difference of people’s interests. Practically, it is really hard for the small exhibition halls to have people’s attention. Even though they spend money more and produce the goods for their exhibitions, there will not be that many people who will pay extra money to buy or rent the goods.

 We thought that people who are trying to open small exhibition and students who are preparing an exhibition doesn’t want the extra income but the interests of people. We wanted to provide them more chances for the amateur artists to introduce themselves and to appeal their works to people. Therefore, we want to fulfill their requirements through “SMART BROCHURE.” Artists can provide good service to their audiences with less cost, and audiences don’t have to pay extra money and get the chance to use many services that artists want to provide, only by installing an application. 
Though there is a similar application, named “Jeonsi bogo,” this service is only for the big exhibitions, so we still need to develop new system.\\\\\\\\\\\\\\\\\\\\\\\\\\\\\\\\\\\\\\\\\\\\\\\\\\\\\\\\\\\\\\\\\\\\\\\\\\\\\\\\\\\\\\\\\\\\\\\\\\
% You must have at least 2 lines in the paragraph with the drop letter
% (should never be an issue)

\section{Index}

I. Introduction\\


II. Requirement

\quad2.1 The Environment for Using Beacon

\quad2.1-1 Bluetooth 2.1-2 OS

\quad2.2 Server requirement

\quad2.2-1 How to send information 2.2-2 Need for back-end server

\quad2.3 User requirement 2.3-1 Accuracy

\quad2.3-2 Unimportant data 2.3-3 Push alarm

\quad2.3-4 Exhibition list

\quad2.3-5 Map

\quad2.3-6 Location of work the exhibition 2.3-7 Work button

\quad2.3-8 Work picture

\quad2.3-9 Text box1

\quad2.3-10 Text box2

\quad2.3-11 Voice button

\quad2.3-12 Previous exhibition

\quad2.3-13 Delete function

\quad2.3-14 Reporting problem

\quad2.4 Customer requirement\\

III. Develop environment

\quad3.1 Choice of software development platform

\quad3.1-1 Which platform and why?

\quad3.1-2 Which programming language and why? 3.2 A cost estimation

\quad3.3 Information of development environment

\quad3.4 Using commercial cloud platform\\

IV. Specification

\quad4.1 Modeling for Specifications 

\quad4.2 Prototype for Specifications

\quad4.3 Specifications for front-end application pages

\quad\quad-BLE seraching outside the application

\quad\quad-My history page

\quad\quad-Information page

\quad\quad-Searching beacon page

\quad\quad-More page

\quad4.4 Specifications for Server

\quad\quad-Filezilla 

\quad\quad-pgadmin \\

V. Architecture Design and Implementation\\

VI. Use Cases
\\\\\\\\\\
\\\\\\\\\\\\\\\\\\\\\\\\\\\\\\\\\\\\\\\\\\\\\\\\\\\\\\\\\\\\\\\\\\\\\\\\\\\\\\\\\\\\
\section{Requirements}

\subsection{Requirement for The Environment for Using Beacon}

1. Bluetooth module

- Users need to have Bluetooth 4.0 or higher module.

2. Smartphone OS

1) ios 7 or higher

2) Android 4.3 or Higher

3) OSX mavericks 10.9\\\\

\subsection{Requirement for Server}
1. How to send information\\

1) The beacon installed in the gallery will send the id code to the smartphone which has the application, and the smartphone will send that id code and the customer information to the server. \\

2) Lastly, the server will send the appropriate data, which is decided by combining the gallery’s information and the customer information, to the smartphone.\\

2.  Needs\\
 We need to develop a back-end server that stores the information and judge the id code sent by beacon.\\\\

\subsection{Requirement for User}
1. Accuracy : Users want more precise sensor when they use beacon technology\\

2. Unimportant data : Users don’t want information which is not necessary\\

3. Push alarm

\quad 1) Push alarm is popped up on the user’s smartphone when passes by an exhibition.

\quad 2) The user can get some information by push the ‘yes’ button.\\

4. Exhibition list 

\quad - If push the button, you can see list of exhibitions you watched. Latest exhibition is located in top of the list.\\

5. Map

\quad 1) Map provides a course how to see the exhibition.

\quad 2) If users push one of the mutton, Smart Brochure gives users the map of the exhibition.\\

6. Location of work in the exhibition

\quad - In the map, users can see some buttons which indicates work name and where works are.\\

7. Work button

\quad - If users push button on the map, they can get screen which has information about the work.\\

8. Work picture

\quad - In the information screen, picture of the work is located left-top. 

\quad - Users can check on whether explanation corresponds to the work by picture.\\

9. Text box1

\quad - Text box1 is located next to work picture. There are work name, artist name, and techniques in the box.\\

10. Text box2

\quad - Text box2 has explanation of the work. If explanation is so long, uses can use scroll technique.\\

11. Voice button

\quad - In the text box2, users can use voice button. If users push the button, they can hear explanation of the work.\\

12. Previous exhibition

\quad - This application can saved data about previous exhibition.\\

13. Delete function

\quad - If users want to delete previous exhibition information, they can delete the data.\\

14. Reporting problem

\quad - When use Smart Brochure, users can find some problem. In this situation, users can report this problem to developer.


\section{Development Environment}

\subsection{Choice of software development platform}
1. Which platform and why? (e.g., Windows, Linux, Web, or etc.) 

1) Windows for android application developing

2) Mac OS for iOS application developing.

2. Which programming language and why?

1) java for android application developing \\

Java is a general-purpose computer programming language that is concurrent, class-based, object-oriented, and specifically designed to have as few implementation dependencies as possible. As of 2015, Java is one of the most popular programming languages in use, particularly for client-server web applications, with a reported 9 million developers. Java was originally developed by James Goslingat Sun Microsystems (which has since been acquired by Oracle Corporation) and released in 1995 as a core component of Sun Microsystems' Java platform. The language derives much of its syntax from C and C++, but it has fewer low-level facilities than either of them.\\


2) objective-C and Swift for iOS application developing.\\
 Objective-C is a general-purpose, object-oriented programming language that adds Smalltalk-style messaging to the C-programming language. It is the main programming language used by Apple for the OS X and iOS operating systems, and their respective application programming interfaces (APIs), Cocoa and Cocoa Touch.
Objective-C's features often allow for flexible, and often easy, solutions to programming issues. Delegating methods to other objects and remote invocation can be easily implemented using categories and message forwarding. Swizzling of the isa pointer allows for classes to change at runtime. Typically used for debugging where freed objects are swizzled into zombie objects whose only purpose is to report an error when someone calls them. Swizzling was also used in Enterprise Objects Framework to create database faults. Swizzling is used today by Apple?s Foundation Framework to implement Key-Value Observing.\\

Swift is a multi paradigm, compiled 
programming language created by Apple Inc. for iOS and OS X development Swift is designed to work with Apple's Cocoa and 
Cocoa Touch frameworks and the large body of existing Objective-C code written for Apple products. Swift is intended to be more resilient to erroneous code ("safer") than Objective-C, and also more concise. It is built with the LLVM compiler framework included in Xcode6, and uses the Objective-C runtime, allowing C, Objective-C, C++ and Swift code to run within a single program, but its proprietary nature may hinder Swift's adoption outside the Apple ecosystem. \\



3) JSON\\
JSON is an open standard format that uses human-readable text to transmit data objects consisting of attribute?value pairs. It is used primarily to transmit data between a server and web application, as an alternative to XML. Although originally derived from the JavaScript scripting language, JSON is a language-independent data format.\\\\

\subsection{Provide a cost estimation for your built. \\
(including any purchase of software/hardware)}

1. cost for server : 1 year for free. And after 1year, there will be additional prices. We predict maybe about 1,000 people will use our service, and DAU(Daily Activity User) will be 300 around. So we will use t1micr instance (AWS), and it’s prices are about $30 per month. So maybe there will be additional $360 per year. 

2. cost for beacon : We will use the “RECO” beacon. Reco beacon is authorised by iBeacon. It’s prices are ₩ 229,000 (10 pieces). 

3 cost for developer : \\\\

\subsection{Provide clear information of your development environment.\\ (e.g., version of software, OS version, your computer resources) }
1. iOS develop 

  1) Mac : OS X Yosemite ver 10.10.1

  2) iOS : iOS 8.3\\

2. Android develop

  1) Windows : Windows7 ultimate

  2) Android OS : Android 4.4 kitkat\\\\

\subsection{Using any commercial cloud platform (e.g., Amazon’s EC2) is definitely a BONUS.\\  }
1.   We will use Apache  Web Server because  it is the world's most widely used web server software. As of June 2013, Apache was estimated to serve 54.2percent of all active website and 53.3percent of the top servers across all domains. (The most important reason is we have already apache web server.)\\

Apache Web Server has some features: 
	
\quad- easy and fast customizing using module. 

\quad- It can handle many traffic easily.

\quad- It can control web server more delicately

\quad- It is tested enough, so it is very stable.\\


2. Software in use 

Any existing software or algorithm in use? (doing a similar task as your proposal; provide a proper reference if there is any)\\

1. Trello : \\

\begin{figure}[htbp]
\begin{center}
    \includegraphics[scale=0.5]{img_kanban}
    \caption{Kanban Paradigm} 
\end{center}
\end{figure}

\begin{figure}[htbp]
\begin{center}
    \includegraphics[scale=0.2]{img_trello}
    \caption{trello} 
\end{center}
\end{figure}

Trello is a free web-based project management application. Trello uses the kanban paradigm for managing project. 
Kanban is a scheduling system for lean and just-in-time (JIT) production. \\

Projects are represented by boards, which contain lists (corresponding to task lists). Lists contain cards (corresponding to tasks). Cards are supposed to progress from one list to the next (via drag-and-drop), for instance mirroring the flow of a feature from idea to implementation. Users can be assigned to cards. Users and boards can be grouped into organizations.\\
 




There is a little similar software in Korea, “전시보GO” But this software service is only for big exhibition so small exhibition artist or students can’t use that services. 


3. Task distribution (If you want, you can provide this later at the next phase - design) 

Which member is responsible for what? \\\\\\\\\\\\\\\\\\\\\\\\\\\\





\section{Specifications}
\subsection{Modeling for Specifications \\}


\begin{figure}[htbp]
\begin{center}
    \includegraphics[scale=0.4]{img_01.jpg}
    \caption{basic structure} 
\end{center}
\end{figure}


\quad Beacon sends a specific ID value to the smart phone, when smart phone comes into it?s signal area. Then smart phone application recognizes this ID value and sends this value to server. Server which has this ID value check the location of beacon. After that, server sends the information or data about exhibition to smart phone.\\\\

\subsection{Prototype for Specifications}
\quad Our application is divided into two parts. One is server side with Amazon Web Service EC2 and
Ruby on Rails. The other part is client side acting at the smart phone. Client side, smart phone application, is structured by objective-C (iOS application), and java(Android application)\\\\\\\\\\\\\\\\\\\\\\\\\\\\\\\\\\\\\\

\subsection{Specification for front-end application pages\\}
\quad 0) BLE seraching outside the application\\
\begin{figure}[htbp]
\begin{center}
    \includegraphics[scale=1]{img_BLE}
    \caption{Information Page02} 
\end{center}
\end{figure}

[BLE]

In our application, there is the service class [SearchBLE.java] which is searching the [BLE]. This service searches the BLE, if the bluetooth module on the smartphone is on. BLE(Bluetooth low energy) is a wireless personal area network(PAN) technology. It is designed and marketed for applications in the healthcare, fitness, security, home entertainment industries, and [beacon].\\

\begin{figure}[htbp]
\begin{center}
    \includegraphics[scale=0.35]{img_tempAct}
    \caption{how temporary activity is operating} 
\end{center}
\end{figure}


If user let bluetooth module [on], smartphone will be searching BLE automatically. That is, it is searching beacon signal. If it finds beacon signal, smartphone get [beacon id]. And smartphone sends this beacon id to server, then server sends to smartphone the all data about exhibition which is stored at that beacon id. After, every data about exhibition is downloaded to smartphone. This downloaded data is showed through [Temporary Activity]. \\

The first page has the brief explains for exhibition and the list of artworks (similar with [My History Page02]). Among these artworks, if you pick one, you can show the details of that artworks such like photos and detail explanations (similar with [My History Page 03]).\\

You have to know this [Temporary Activity] is totally different page with main application. This is [only] temporary. This activity cannot access to main application, and also main application cannot access this activity neither. This activity is only for showing the data from the server. And in main application, not every data showed at temporary activity is stored. Only the name of exhibition and the downloaded date are stored. \\\\\\\\

\quad 1) My history page\\

\begin{figure}[htbp]
\begin{center}
    \includegraphics[scale=0.2]{img_My01}
    \caption{BLE} 
\end{center}
\end{figure}


[My History Page01]

This is the first page of application. And you can access this page by tab menu under the display. There is the list of exhibitions which user have already seen.  \\

\begin{figure}[htbp]
\begin{center}
    \includegraphics[scale=0.2]{img_exhiDetail01}
    \caption{My History Page02} 
\end{center}
\end{figure}

[My History Page02] 

This is the detail page of the exhibition01. To access this page, applications has to communicate with the back-end server. Server will send the data of Exhibition 01 which the user already downloaded by beacon communication at that Exhibition. On Navigation bar, there is the name of the exhibition. Under the navigation bar, the left side of the first cell,  there is main image of exhibition. And Next to the main image, the right side of the first cell, there is the brief explanation for this exhibition. There will be the information of this exhibition such like the theme of exhibition, the name of exhibition center, address of exhibition center. Below First cell, there is the list of artwork which is displayed in this exhibition.  user can see the detail information about the artwork such as image of artwork, text explanation of artist, or voice-audio explanation. \\\\\\

\begin{figure}[htbp]
\begin{center}
    \includegraphics[scale=0.2]{img_exhiDetail02}
    \caption{My History Page03} 
\end{center}
\end{figure}

[My History Page03]

This is the detail page of artwork. On the Navigation bar, there is the name of artwork. Under the Navigation bar, there is the Image of the artwork(If user touch the small image, the pop up window will appear, and user can see the big size image). Below the artwork image, there is the play and stop button. This button is for voice-audio explanation. Voice-audio explanation is not for every artwork. We offer the voice-audio explanation only for the artwork that artist want, and artwork that artist offer the voice-audio explanation data. So if there is the voice-audio explanation, there will be play and stop buttons. And if there is no voice-audio explanation, the play and stop buttons will not exist. Under the Image and buttons, there is the text explanations that artist offer. \\\\\\\\\\\\\\\\\\
\quad 2) Information page\\

\begin{figure}[htbp]
\begin{center}
    \includegraphics[scale=0.2]{img_Info01}
    \caption{Information Page01} 
\end{center}
\end{figure}

[Information Page01]

user can access the Information page by touching tab menu button under the screen. When user touch the [Info] button under the display, every data is downloaded from server. This page is for noticing the exhibition. There is the list of the exhibitions which is on going now or which will be started. user can check the detail information about the exhibition that user like by touching the name of the exhibition. \\\\\\\\\\\\\\\\\\\\\\\\\\\\

\begin{figure}[htbp]
\begin{center}
    \includegraphics[scale=0.2]{img_infoDetail}
    \caption{Information Page02} 
\end{center}
\end{figure}

[Information Page02]

This page shows the detail information of the exhibition. user can move to the exhibition list page by back button on the navigation bar. On navigation bar, there is the name of the exhibition. Below the navigation bar, user can find the brief information about the exhibition such as the main image of the exhibition, theme of the exhibition, and map or address of exhibition. \\\\\\\\\\\\\\\\\\\\\\\\\\\\\\\\\\



\quad 3) Setting page

\begin{figure}[htbp]
\begin{center}
    \includegraphics[scale=0.2]{img_setting01}
    \caption{Setting Page} 
\end{center}
\end{figure}

[Setting Page]

In Setting page, there will be the additional function. For example, there will be the on/off button that control the searching BLE. If the user makes the button-state [on], smartphone will search beacon signal and get beacon id ( explained and [0)BLE searching outside the application]). And if user makes the button-state [off], smartphone will not search any beacon signal outside the application. And there will be another board for inform the latest version or notice etc. \\\\\\\\\\\\\\\\\\\\\\\\\\

\subsection{Specifications for Server}

\quad 1) Filezilla \\
We use the [filezilla] to upload the data such like image, and text for exhibition. FileZilla is free-open source cross platform. It consists of filezilla client and filezilla server. It can used at windows, mac os, and linux. \\


\begin{figure}[htbp]
\begin{center}
    \includegraphics[scale=0.2]{img_filezilla}
    \caption{filezilla01} 
\end{center}
\end{figure}


\begin{figure}[htbp]
\begin{center}
    \includegraphics[scale=0.2]{img_filezilla02}
    \caption{filezilla02} 
\end{center}
\end{figure}

If artist want to upload the data of his exhibition, he needs to contact us and send the data to us first. After receiving data from artist, we can upload the exhibition data to our server. Then, data is stored at our web server, and wail the signal calling it.\\\\\\\\\\\\\\

\quad 2) Pgadmin \\

We use [pgadmin] for make and control the database. 
\begin{figure}[htbp]
\begin{center}
    \includegraphics[scale=0.2]{img_pgadmin01}
    \caption{pgadmin} 
\end{center}
\end{figure}

In first picture, there are tables of our database. We made 5 tables.\\

\begin{figure}[htbp]
\begin{center}
    \includegraphics[scale=0.2]{img_pgadmin02}
    \caption{pgadmin table01} 
\end{center}
\end{figure}

First, [bc.exh.device] table is for beacon id [PK](This beacon is at the entrance of the exhibition). At this table, we make the exhibition code and match to beacon id. And we make the exhibition code to.\\\\\\\\\\\\\\\\\\


\begin{figure}[htbp]
\begin{center}
    \includegraphics[scale=0.2]{img_pgadmin002}
    \caption{pgadmin table02} 
\end{center}
\end{figure}

Second, [bc.exhibition] table is for  brief information about exhibition. Exhibition code is  [PK]. This table is connected to [bc.exh.device]table.\\

\begin{figure}[htbp]
\begin{center}
    \includegraphics[scale=0.2]{img_pgadmin003}
    \caption{pgadmin table03} 
\end{center}
\end{figure}

[bc.file] table is for image files. There is the 3 type integers. Type integer 1 is for exhibition main image. It is used at My history page02, and Information page02. Type integer 2 is for thumbnail image of artwork. It is used at My history page 02. Type integer 3 is for the image of artwork. It is used My history page 03.\\\\\\\\\\\\\\


\begin{figure}[htbp]
\begin{center}
    \includegraphics[scale=0.2]{img_pgadmin004}
    \caption{pgadmin table04} 
\end{center}
\end{figure}

[bc.pc.device] table is for beacon id[PK]. But this beacon is located nearby artwork. If user goes to artwork, then application recognize the beacon and send to server, this table. \\

\begin{figure}[htbp]
\begin{center}
    \includegraphics[scale=0.2]{img_pgadmin005}
    \caption{pgadmin table05} 
\end{center}
\end{figure}

[bc.piece] table is for detail information about artwork. Artwork has its own code[PK]. \\\\\\\\\\\\\\\\\\\\

\section{Architecture Design and Implementation}
\subsection{Overall architecture}
\begin{figure}[htbp]
\begin{center}
    \includegraphics[scale=0.5]{img_archi}
    \caption{Overall architecture} 
\end{center}
\end{figure}
\subsection{Directory organization}
\begin{figure}[htbp]
\begin{center}
    \includegraphics[scale=0.5]{img_direc}
    \caption{Directory organization} 
\end{center}
\end{figure}
\quad
 \\\\\\\\\\\\\\\\\\\\\\\\\\\\\\\\\\\\\\\\\\\\\\\\\\\\\\\\\\\\\\\\\\\\\\\\\\\\\\\\\\\\\\\\\\\\\\\\\
\subsection{Code analysis}
\quad 1) Splash\\\\
-purpose : To show a cover page during the program is loading on the device. \\
\\ -functionality : Splash is a kind of loading screen. The function of splash is gaining time to get data which is necessary for program. Additionally, we can promote our application name during splash time.\\
\\ -location of source code : SmartBrochure/appsrc/java/com/example/jay/smart\_broch
ure/Splash.java\\
\\ -class components\\
\\ a. 'onCreate' method is implemented firstly in the activity class like main() method in JavaSE. In the onCreate, setContentView shows basic layout which is used to compose main page. By using postDelayed in handler, this method can do function how long loading page appear. \\
\\ b. onCreateOptionsMenu method does initializing option menu of activity.\\
\\c. In onOptionsItemSelected method, Handle action bar item clicks here. The action bar will automatically handle clicks on the Home/Up button, so long as you specify a parent activity in AndroidManifest.xml.\\
\\ -how/why you used it : If you want to implement some program, the program needs time to get source which is vital for program. Splash does function which gains time getting data.\\\\
\quad 2) SearchBle\\\\
-purpose : To search beacon which uses Bluetooth Low Energy(BLE). \\
\\ -functionality : SearchBLE is implemented on application background. Usually almost classes inherit Activity class, but SearchBLE inherits Service class because searching function isn’t seen in the smartphone screen but implemented background. This service searches the BLE devices around the user and check if the BLE address is in our database or not. If the address is in our database and it is not the one that user already visited, this service will send the BLE address to the server to get the information of the exhibition and give the push alarm to the user to get the brochure.\\
\\ -location of source code : SmartBrochure/appsrc/java/com/example/jay/smart\_broch
ure/SearchBle.java\\
\\ -class components\\
\\ a. BroadcastReceiver uses the onReceive() method to get the state of the Bluetooth. If the devices’ Bluetooth is on, this starts the Searching BLE service, by using Timer instance, which makes the period of searching time and so on. If the Bluetooth becomes off, this stops searching BLE devices by calling cancel() method in the Timer instance and scanLeDevice() method by giving the parameter as “false”. It is used to realize operation when application completed specific task.\\
\\ b. Timer is basic class in Android. By using Timer class, we can set schedule when searching starts and how long search beacon. When we use Timer instance, Android appreciate '1 = 1 millisecond'. Therefore, if we want to set searching time 3 seconds, we have to put value 3*1000. In our code, we set it as schedule(search, 10*1000, 20*1000); which meanse it will start searching after 10 secs, and for 20secs, and it will rotate.\\
\\ c. onLeScan's main function is using Handler, which means it will use the Threads. In the run() method, which is working on the thread, the application will check the database if there is a valid address that our service is using, and if two are matched, this will check whether the user has visited to this exhibition by searching the History table in the Database class. Only when there is no history about the beacon address, it will send the address and the code to the server to get the information about the exhibition and will give the device push-alarm which will help the user to get the brochure. When we search beacon, there are so many beacons even if we don't need. If we don’t figure out whether it is valid or not, we have to send all beacon data which are not necessary to server. That can provoke wasting of data use. By using this method, we can send appropriate beacon data to server. Additionally, if an appropriate beacon data sent to server, the method will save the address in the temporary ArrayList<String> variable to prevent sending same address to the server over and over again. If we don’t do this, push alarm will come continuously while audiences watch the exhibition.\\
\\ d. sendId() method is the core part of this service which sends the data to the server and get the data from server to the application. Using HashMap<String, Object> type of variable, we put the beacon id and make it to JSONObject, which server wants as the data type to get from the device. After that, by using DefaultHttpClient, the device will get the data from the server as JSONArray type, so we will cast the datatype that we want to use, and make the push alarm by using NotificationCompat class.\\
\\ -how/why you used it : By using  Auto\_Start class, we made SearchBLE service start when the devices’ booting is completed, only if the auto-search option is on. Therefore, if the user set up the option as on, the user doesn’t need to turn on the application to get the brochure. By only turning on the Bluetooth, this service will search the beacons around the user and send the data to the server and will notify the user by push alarm. This is one of the core concept of IoT..\\\\
\quad 3) MainActivity\\\\
-purpose : To show main page and constitute tab view which consist of My, Info, and Setting for customer.\\
\\ -functionality : MainActivity is the body of SmartBrochure. Most classes are used in this class. The main function is showing main page which is made up of several components.\\
\\ -location of source code : SmartBrochure/appsrc/java/com/example/jay/smart\_broch
ure/MainActivity.java\\
\\ -class components\\
\\ a. turnOnBT method instructs whether to turn on or turn off Bluetooth to customers by dialogue. If Bluetooth is turned on before starting application, notification dialogue will not occur on the device.\\
\\ b. getServiceTaskName checks if the SearchBLE service is now running or not. If it is not running, we will start SearchBLE service in order to search the beacons if the Bluetooth is on.\\
\\ c. We overrode onDestroy() method.  When the application destroys, we need to turn of the SearchBLE service too, if the auto-searching option is off. Therefore, we opened database in this method to check if the option is turn on or off.\\
\\ -how/why you used it : MainActivity is foundation of application. We build the tab activity to make the tabs for the pages, My, Info, Setting. We made this class inherit TabActivity, but we will fix this to FragmentActivity because TabActivity is deprecated.\\\\
\quad 4) My\\\\
-purpose : My\_Clicked shows the list of art work. The list includes name and image of each art work.\\
\\ -functionality : First tab button on the main page and if you execute SmartBrochure, you can see my page firstly. If you click one of the list, list of the exhibition work appears.\\
\\ -location of source code : SmartBrochure/appsrc/java/com/example/jay/smart\_broch
ure/My.java\\
\\ -class components\\
\\ a. sendId method does the same function as the sendId method in the SearchBLE. We get the exhibition code and the BLE address that is already saved in the Database when the user visitied the exhibition, and send them to the server to get the exhibition brochure from the server. After server gets the beacon data and the valid exhibition code, the servier will send the exhibition data (exhibition name, list of work, image and so on) to SmartBrochure and we go to the My\_Clicked class to show the brochure.\\
\\ b. customAdapter's main function is giving data to list view. In the customAdapter, getView method put the data to the list.\\
\\ c. We made this listview get the OnItemClickListner, so when user touch the list, the page will go to the My\_Clicked page which gets the data of the exhibition.\\
\\ -how/why you used it : My page gets the list of history from the Database and put them to the listview. We thought that listview is the best way to show the list to the user. Between My and My\_Clicked, the data communication between the device and server occurs.\\\\
\quad 5) My\_Clicked\\\\
-purpose : My\_Clicked shows the list of art work of the exhibition that the user visited before. Most of the function is as same as Push\_Clicked.\\
\\ -functionality : My\_Clicked class must need to communicate with server in order to get the brochure from the server. In the My page we sent the id and exhibition code to the server, we will get the brochure data from the server in My\_Clicked page and will show the brochure.\\
\\ -location of source code : SmartBrochure/appsrc/java/com/example/jay/smart\_broch
ure/My\_Clicked.java\\
\\ -class components\\
\\ a. ImageLoaderConfigurationbition is used to show the image from the image URL we get from the server. In the application, image file is not anywhere. However, in order to use work image, we can use server. When we need some image file, we get URL which has image from server. In other words, just borrow image file when we need. If we go out of the page which has image, image file is removed from application.\\
\\ b. We use customAdapter and OnClickListener, and so on, which is used in other pages also.\\
\\ -how/why you used it : We communicate with the server here again, because we don’t want to save so many imagefiles to the application database, which will cause some problems of the storage of the device. Instead, we chose to communicate with the sever again, so we gets all of the data we use here from the server.\\\\
\quad 6) Push\_Clicked\\\\
-purpose : Push\_Clicked also shows the list of art work. The list includes name and image of each art work.\\
\\ -functionality : Push\_Clicked class must need to communicate with server in order to get the brochure from the server. In the Push page we sent the to the server, we will get the exhibition code and brochure data from the server in Push\_Clicked page and will show the brochure when user clicked the push alarm button which was sent before.\\
\\ -location of source code : SmartBrochure/appsrc/java/com/example/jay/smart\_broch
ure/Push\_Clicked.java\\
\\ -class components\\
\\ a. ImageLoaderConfigurationbition is used to show the image from the image URL we get from the server. In the application, image file is not anywhere. However, in order to use work image, we can use server. When we need some image file, we get URL which has image from server. The image URL or the Image file that we used here will never be stored in the database.\\
\\ b. In the onCreate method, we get the data of the brochure. Here, we need to save the beacon’s id and exhibition code in the database in order to keep the history of user’s exhibiting. We used History class to save the exhibition title, name, and the BLE address. After that, we use the addHistory method to save it in the database with SQLite.\\
\\ -how/why you used it : Push\_Clicked is very similar with My\_Clicked. However, they are different in sending id value to server. My\_Clicked is focused on existing exhibition information. In contrast, Push\_Clicked is focused on new information from beacon search. While My\_Clicked page sent the beacon’s id and exhibition code, in Push\_Clicked page, we send only the beacon’s id that SearchBLE service has searched to the server. \\\\
\quad 7) Explanation\\\\
-purpose : Explanation class has function which gives the detailed description about art work. Users can read the detailed image and the explanation of the specific work that the user wants to see the detail.\\
\\ -functionality : When you click one of the art works in the list(in the My\_clicked or Push\_Clicked), Explanation class is implemented. In this process, communication with server is very essential like My\_clicked or Push\_Clicked. Explanation also use image by using URL from server.\\
\\ -location of source code : SmartBrochure/appsrc/java/com/example/jay/smart\_broch
ure/Explanation.java\\
\\ -class components\\
\\ a. In the Explanation class, onCreate method's main function is setting ImageView and TextView which will be used after getting data from the server. They are given from server. At the end of this method, we call the setExplanation() method to set the data.\\
\\ b. setExplanation method uses some function like DisplayImageOptions and ImageLoaderConfiguration. By using them, we set the image from the imageURL, and the explanation of the artwork.\\
\\ -how/why you used it : Explanation class is the most informative section in our application. Like My\_Clicked, we use server to use adequate image, because image data is very important in exhibition information. \\\\
\quad 8) Info\\\\
-purpose : Info notifies several exhibitions’ information which is what is exhibiting now.\\
\\ -functionality : Second tab button on the main page so if you execute SmartBrochure, you can't see Info page firstly. If you click Info button, list of several exhibition which are exhibition now appears. We get this information from the server, which means, if the user click on the Info tab button, the device will communicate with the server to get the information.\\
\\ -location of source code : SmartBrochure/appsrc/java/com/example/jay/smart\_broch
ure/Info.java\\
\\ -class components\\
\\ a. In the Info class, sendId method is the important method because uses server like several other classes. Here, the device will just send the url Id and then the server will send the list of the exhibitions only.\\
\\ b. Here, we also used ListView to show the list of the exhibitions which users can go to and get the brochure and see the artworks now.\\
\\ -how/why you used it : This is just a basic page of ListView, to show the list of exhibitions our service is providing. We will get the list from the server when user tries to go to the info page. If the Internet is not on on the device, this page will show nothing beacause it will not get any data from the server. The reason why we made this page is that we thought that users might need the information of the exhibitions and can choose the exhibition that they want to go. They will choose one or several of the exhibitions from the list and when they go to the exhibitions, they can get the brochure on their device.\\\\
\quad 9) Info\_Clicked\\\\
-purpose : Info\_Clicked shows exhibition name, location, date and brief explanation.\\
\\ -functionality : Info\_Clicked class also must need communication with server. If application sends the id value which is in the database, after that server sends information about the exhibition which is correct with id to application.\\
\\ -location of source code : SmartBrochure/appsrc/java/com/example/jay/smart\_broch
ure/Info\_Clicked.java\\
\\ -class components\\
\\ a. In setInformation, there are so many function to make Info\_Clicked class. To show title, explanation, date and address, use text view.  ImageLoaderConfigurationbition is also used for implanting image like My\_Clicked. In the application, image file doesn't store anywhere. However, in order to use work image, we can use server. When we need some image file, we get URL which has image from server. In other words, just borrow image file when we need. If we go out of the page which has image, image file is removed from application.\\
\\ -how/why you used it : Info\_Clicked knows us information about exhibition that we can go. People who are interested in art want to kwon exhibition information which they can go nowadays. Also we add some data like exhibition image, brief explanation and so on. These information will satisfy user's needs.\\\\
\quad 10) Setting\\\\
-purpose : Setting is used to set automatic search by using Bluetooth and show the version of the application.\\
\\ -functionality : By setting, users can choose whether searching the beacon devices automatically or not. If you want to get exhibition data when you hang around outside, ust set 'automatic search' on in the setting.\\
\\ -location of source code : SmartBrochure/appsrc/java/com/example/jay/smart\_broch
ure/Setting.java\\
\\ -class components\\
\\ a. In the onCreate method, auto\_search find signal from beacon which is around user. btn\_onoff set that user turn on/off the automatic search function. At first, we get the on/off data from the database to show the user the current setting.\\
\\ b. In the onCreate method, auto\_search find signal from beacon which is around user. btn\_onoff set that user turn on/off the automatic search function. At first, we get the on/off data from the database to show the user the current setting.\\
\\ -how/why you used it : If user always turn on Bluetooth and turn on auto\_search function, user's smartphone battery will be consumed rapidly. Therefore, we thought it is very important to set these functions on/off. This function can save users’ smartphone battery and satisfy the users’ preference.\\\\
\quad 11) Database\\\\
-purpose : To store the lists of beacons that our service is using/ will use, the history of the user’s own exhibitions that he/she has visited and got the brochure from it, and save the setting of automatic beacon searching preference.\\
\\ -functionality : In the services and pages of our service, especially SearchBLE, My, and Setting, we need to use the database to keep the user’s preferences and the history of visiting the exhibitions.\\
\\ -location of source code : SmartBrochure/appsrc/java/com/example/jay/smart\_broch
ure/Database.java\\
\\ -class components\\
\\ a. First, the database class inherits SQLiteOpenHelper to use the database with SQLite.\\
\\ b. setExplanation method uses some function like DisplayImageOptions and ImageLoaderConfiguration. By using them, we set the image from the imageURL, and the explanation of the artwork.\\
\\ c. init() method - when the user first downloaded this application, the device doesn’t have any database, but we need the list of the beacons that we use. Therefore, when the application was on for the first time, call init() method in Database.java and this will make the list of the beacons and set the ONOFF preference “0”, which means “off”, as default.\\
\\ d. getOnoff() method - search the CheckOnOff table and get ONOFF row and return in order to check if the onoff setting is on or off.\\
\\ e. editOnoff() method - get String value as the parameter and edit the ONOFF row in the CheckOnOff table. \\
\\ f. getList() method - return the list of the S-Brochure table.\\
\\ g. updateBeacons() method - using Cursor class and ContentValues class, update the beacons that out service use.\\
\\ h. getBeaconse() method - return the list of the beacons that we are using, to identify the BLE address that is valid for this application.\\
\\ i. addHistory() method - using the History class, add the beacon’s address and the name and code of the exhibition that user visited to the S\_Brochure table.\\
\\ j. getHistoryName(), getHistoryAddress(), and getHistoryCode() methods - three of these methods searches the DB and return the beacon’s address and the name and code of the exhibition that user visited with ArrayList<String> type.\\
\\ k. searchHistory() method - with the parameter of String type, search the DB to find if there is the same address in the DB.\\
\\ -how/why you used it : We developed this database with SQLite. If some classes(activities) needs the database information, they can instantiate the Database class and use the methods that we have made so that they will be able to access to the Database and get the information they need and edit the database if it is needed. The database is essential to our Smart Brochure service, since we need to manage the history of the user’s visiting exhibitions; if they want to read the previous brochure, they don’t need to go to the exhibition again if they have already visited. Also, the application needs to have the list of the beacons that it should search and send the information to the server, because there are a lot of BLE devices around us. Therefore, we had to give the application the list of beacons that it should send the information if one of them is sensed. Lastly, we have the function of auto-searching setting, so the database has the setting and can figure if the setting is on or off.\\\\
\quad 12) History\\\\
-purpose : To make it easier and flexible to keep the database.\\
\\ -functionality : There are several methods to get the data and give the date to the caller. Other classes will instantiate this class and save data in the object, and will send this object to the database.\\
\\ -location of source code : SmartBrochure/appsrc/java/com/example/jay/smart\_broch
ure/History.java\\
\\ -class components\\
\\ a. This class has getAddress(), setAddress(), getName(), setName(), getCode(), setCode() method to make the flexibility higher.\\
\\ -how/why you used it : We just made some methods to get and set the data to and from the database. We will use this class by making it instance in other classes. Save the data in the instance and transfer the instance to the Database. Without this method, it becomes more complex to keep the database.\\\\.\\\\\\\\\\\\\\\\\\\\\\\\\\\\\\\\\\\\\\\\\\\\\\\\\\\\\\\\\\\\\\\\\\\\\\\\\\\\\\\\\\\\\\\\\\\\\\\\\\\\\\\\\\\\\\\\\\\\\\\\\\\\\\\\\\\\\\\\\\\\\\\\\\\\\\\\\\\\\\\\\\\\\\\\\\\\\\\\\\\\\\

\section{Use Cases}

\begin{figure}[htbp]
\begin{center}
    \includegraphics[scale=0.75]{img_flowchart}
    \caption{Application Flowchart} 
\end{center}
\end{figure}



.\\\\\\\\\\\\\\\\\\\\\\\\\\\\\\\\\\\\\\\\\\\\\\\\\\\\\\\\\\\\\\\\\\\\\\\\\\\\\\\\\\\\\\\\\\\\\\\\\\\\\\\\\\\\\\\\\\\\\\\\\\\\\\\\\\

\begin{figure}[htbp]
\begin{center}
    \includegraphics[scale=0.5]{img_serverFlowchart}
    \caption{Server Flowchart} 
\end{center}
\end{figure}

.\\\\\\\\\\\\\\\\\\\\\\\\\\\\\\\\\\\\\\\\\\\\\\\\\\\\\\\\\\\\\\\\\\\\\\\\\\\\\\\\\\\\\\\\\\\\\\\\\\\\\\\\\\\\\\\\\\\\\\\\\\\\\\\\\\
\quad6-1) BLE searching and push notice
\begin{figure}[htbp]
\begin{center}
    \includegraphics[scale=0.2]{img_capture01}
    \caption{BLE searching and push notice} 
\end{center}
\end{figure}\\
\quad If User turn on the bluetooth module, smartphone finds the BLE signal, and then Server sends to notice that information data about Exhibition is downloaded. If user touch the push notice, then temporary activity is open.\\\\\\\\\\\

\quad6-2) Temporary Activity01 \\
\begin{figure}[htbp]
\begin{center}
    \includegraphics[scale=0.2]{img_capture02}
    \caption{Temporary Activity 01} 
\end{center}
\end{figure}\\
\quad This is the page when temporary page is open. It shows the information received from server. Upper-side of the page, there are the name and main image of exhibition. Under, there are the artwork lists which are displayed at the exhibition now user is seeing.\\\\\\\\\\\\\\

\quad6-2) Temporary Activity02 \\
\quad If the user touch one artwork, he can watch the detail of artwork. Of course every data is received from server. There are image, name, and detail descriptions of the artwork. 
\begin{figure}[htbp]
\begin{center}
    \includegraphics[scale=0.18]{img_capture03}
    \caption{Temporary Activity02} 
\end{center}
\end{figure}
\\

\quad6-3) Main Application - My History Page01\\
\begin{figure}[htbp]
\begin{center}
    \includegraphics[scale=0.2]{img_capture04}
    \caption{My History01} 
\end{center}
\end{figure}\\
\quad User will see this page when he run the application. It is the first page of main application. First, there is tab menu under the screen. By this tab menu, user can move to page he want to see.  This page is for user own. The data which he received from server and saw through temporary activity is stacked at this page. But this page only store the name of exhibition. Because the memory of smartphone. If user touch some exhibition which want to see again, then application send to the server the [eh.code] and get the data of that exhibition.\\\\\\\\\\\\\\\

\quad6-3) Main Application - My History Page02\\
\begin{figure}[htbp]
\begin{center}
    \includegraphics[scale=0.2]{img_capture05}
    \caption{My History02} 
\end{center}
\end{figure}\\
\quad This page shows the detail information about the exhibition. User can see the brief details about the exhibition theme, the Artist, and the list of artwork which is displayed on that exhibition. \\\\

\quad6-3) Main Application - My History Page03\\
\begin{figure}[htbp]
\begin{center}
    \includegraphics[scale=0.2]{img_capture06}
    \caption{My History03} 
\end{center}
\end{figure}\\
\quad If user touch any artwork at the My History Page01, user can see the detail information about that artwork like this page.\\\\\\\\

\quad6-3) Main Application - Information Page01
\begin{figure}[htbp]
\begin{center}
    \includegraphics[scale=0.2]{img_capture07}
    \caption{Information Page01} 
\end{center}
\end{figure}

This is the second menu page of the application. User can move to this page from the other using tab menu under the screen. This page is for inform about the other exhibition. On the screen, there will be the list of exhibition that user has not seen yet. If user touch the INFO menu from other page, or swipe down the list, then application connect to the server([bc.exhibition]) and get the data of the list of the exhibition. It means, user can refresh the list whenever he wants.\\\\\\\\\\\\\\\\

\quad6-3) Main Application - Information Page02
\begin{figure}[htbp]
\begin{center}
    \includegraphics[scale=0.2]{img_capture08}
    \caption{Information Page02} 
\end{center}
\end{figure}

If user select one exhibition, he can see the detail information about that exhibition such like the main image of the exhibition, information about the artist, the address of the exhibition and so on. \\\\\\\\\\\\\\\\\\\\\\\\\\\\\\\\\\\\\\\\\\

\quad6-3) Main Application - Setting Page
\begin{figure}[htbp]
\begin{center}
    \includegraphics[scale=0.2]{img_capture09}
    \caption{Setting Page01} 
\end{center}
\end{figure}

User can turn on or off the function searching the BLE. - Turn on\\\\\\\\\\\\\\\\\\\\\\\\\\\\\\\\\\\\\\\\\\\\\\\\

\begin{figure}[htbp]
\begin{center}
    \includegraphics[scale=0.2]{img_capture10}
    \caption{Setting Page02} 
\end{center}
\end{figure}

User can turn on or off the function searching the BLE. - Turn off\\\\\\\\
.
% An example of a floating figure using the graphicx package.
% Note that \label must occur AFTER (or within) \caption.
% For figures, \caption should occur after the \includegraphics.
% Note that IEEEtran v1.7 and later has special internal code that
% is designed to preserve the operation of \label within \caption
% even when the captionsoff option is in effect. However, because
% of issues like this, it may be the safest practice to put all your
% \label just after \caption rather than within \caption{}.
%
% Reminder: the "draftcls" or "draftclsnofoot", not "draft", class
% option should be used if it is desired that the figures are to be
% displayed while in draft mode.
%
%\begin{figure}[!t]
%\centering
%\includegraphics[width=2.5in]{myfigure}
% where an .eps filename suffix will be assumed under latex, 
% and a .pdf suffix will be assumed for pdflatex; or what has been declared
% via \DeclareGraphicsExtensions.
%\caption{Simulation Results}
%\label{fig_sim}
%\end{figure}

% Note that IEEE typically puts floats only at the top, even when this
% results in a large percentage of a column being occupied by floats.


% An example of a double column floating figure using two subfigures.
% (The subfig.sty package must be loaded for this to work.)
% The subfigure \label commands are set within each subfloat command, the
% \label for the overall figure must come after \caption.
% \hfil must be used as a separator to get equal spacing.
% The subfigure.sty package works much the same way, except \subfigure is
% used instead of \subfloat.
%
%\begin{figure*}[!t]
%\centerline{\subfloat[Case I]\includegraphics[width=2.5in]{subfigcase1}%
%\label{fig_first_case}}
%\hfil
%\subfloat[Case II]{\includegraphics[width=2.5in]{subfigcase2}%
%\label{fig_second_case}}}
%\caption{Simulation results}
%\label{fig_sim}
%\end{figure*}
%
% Note that often IEEE papers with subfigures do not employ subfigure
% captions (using the optional argument to \subfloat), but instead will
% reference/describe all of them (a), (b), etc., within the main caption.


% An example of a floating table. Note that, for IEEE style tables, the 
% \caption command should come BEFORE the table. Table text will default to
% \footnotesize as IEEE normally uses this smaller font for tables.
% The \label must come after \caption as always.
%
%\begin{table}[!t]
%% increase table row spacing, adjust to taste
%\renewcommand{\arraystretch}{1.3}
% if using array.sty, it might be a good idea to tweak the value of
% \extrarowheight as needed to properly center the text within the cells
%\caption{An Example of a Table}
%\label{table_example}
%\centering
%% Some packages, such as MDW tools, offer better commands for making tables
%% than the plain LaTeX2e tabular which is used here.
%\begin{tabular}{|c||c|}
%\hline
%One & Two\\
%\hline
%Three & Four\\
%\hline
%\end{tabular}
%\end{table}


% Note that IEEE does not put floats in the very first column - or typically
% anywhere on the first page for that matter. Also, in-text middle ("here")
% positioning is not used. Most IEEE journals/conferences use top floats
% exclusively. Note that, LaTeX2e, unlike IEEE journals/conferences, places
% footnotes above bottom floats. This can be corrected via the \fnbelowfloat
% command of the stfloats package.





% conference papers do not normally have an appendix


% use section* for acknowledgem



% trigger a \newpage just before the given reference
% number - used to balance the columns on the last page
% adjust value as needed - may need to be readjusted if
% the document is modified later
%\IEEEtriggeratref{8}
% The "triggered" command can be changed if desired:
%\IEEEtriggercmd{\enlargethispage{-5in}}

% references section

% can use a bibliography generated by BibTeX as a .bbl file
% BibTeX documentation can be easily obtained at:
% http://www.ctan.org/tex-archive/biblio/bibtex/contrib/doc/
% The IEEEtran BibTeX style support page is at:
% http://www.michaelshell.org/tex/ieeetran/bibtex/
%\bibliographystyle{IEEEtran}
% argument is your BibTeX string definitions and bibliography database(s)
%\bibliography{IEEEabrv,../bib/paper}
%
% <OR> manually copy in the resultant .bbl file
% set second argument of \begin to the number of references
% (used to reserve space for the reference number labels box)




% that's all folks
\end{document}


